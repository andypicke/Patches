\documentclass[11pt]{article}
\usepackage{geometry}                % See geometry.pdf to learn the layout options. There are lots.
\geometry{letterpaper}                   % ... or a4paper or a5paper or ... 
%\geometry{landscape}                % Activate for for rotated page geometry
%\usepackage[parfill]{parskip}    % Activate to begin paragraphs with an empty line rather than an indent
\usepackage{graphicx}
\usepackage{amssymb}
\usepackage{amsmath}
\usepackage{epstopdf}
\usepackage{hyperref}
\DeclareGraphicsRule{.tif}{png}{.png}{`convert #1 `dirname #1`/`basename #1 .tif`.png}



\graphicspath{
{/Users/Andy/Cruises_Research/ResearchNotes/}
}

\title{Research Notes}
\author{Andy Pickering}
%\date{}                                           % Activate to display a given date or no date



\begin{document}
\maketitle

\tableofcontents
\newpage



%\begin{figure}[htbp]
%\includegraphics[scale=0.8]{}
%\caption{}
%\label{}
%\end{figure}



\clearpage
%~~~~~~~~~~~~~~~~~~~~~~~~~~~~~~~~~~~~~~
\section{Feb 11, 2017}

So after correcting (I think) $N^2$, it appears that we don't get gammas near 0.2 after all.... Trying now to summarize what i've done:

\begin{itemize}
\item Identify patches (overturns) in Chameleon profiles using potential temperature. \verb+FindPatches_EQ14_Raw.m+
\item Compute $N^2$, $T_z$, $\chi$, and $\epsilon$ for each patch. $N^2$ and $T_z$ are computed with two different methods. \verb+Compute_N2_dTdz_ChamProfiles_V2.m+
\item `line' method: fit line to T,density to get slope.
\item `bulk' method: From Smyth et al. $N^2$ is computed from the bulk $T_z$ using a linear fit of density to temperature; $\sigma=\alpha \theta$, and then $N^2=-\frac{g}{\rho_o} \alpha T_{z <bulk>}$
 (Fig \ref{sgth_theta_fit}). \verb+Fit_rho_vs_T.m+.
\item Compute $\Gamma=\frac{N^2\chi}{2\epsilon T_{z}^{2}}$ from above values (Fig \ref{gamshist}).
\end{itemize}


\begin{figure}[htbp]
\includegraphics[scale=0.8]{eq14_sght_vs_theta_fit.png}
\caption{Fit of potential density to potential temperature for EQ14 chameleon profiles.}
\label{sgth_theta_fit}
\end{figure}



\begin{figure}[htbp]
\includegraphics[scale=0.8]{eq14_gamma_hist_line_bulk.png}
\caption{$log_{10}[\Gamma$] for EQ14 patches. For (1) 1m binned data (2) `bulk' $N^2$ and $T_z$ (3) `line' $N^2$ and $T_z$. Dashed line shows $\Gamma=0.2$}
\label{gamshist}
\end{figure}



\end{document}  